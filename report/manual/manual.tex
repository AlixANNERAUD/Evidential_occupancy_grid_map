
\chapter{Manual}

\section{Build}

\subsection{Prerequisites}

First of all, you will need the following software installed:
\begin{itemize}
    \item \textbf{ROS Melodic} : You can install it by following the instructions on the official website: \url{http://wiki.ros.org/melodic/Installation}. You likely need a Ubuntu 18.04 (Bionic Beaver) system.
    \item \textbf{Octomap} : Check the package page on the ROS wiki for installation instructions: \url{http://wiki.ros.org/octomap}.
\end{itemize}

\subsection{Build}

Clone the repository from GitHub:
\begin{minted}{bash}
git clone https://github.com/DimitriTimoz/tiled-evidential-occupancy-grid
\end{minted}

To build the project, run the following commands in the root directory of the project:
\begin{minted}{bash}
catkin_make install
\end{minted}

\section{Run}

After starting the ROS master (with the command \texttt{roscore}), you can run the project with the following commands in the root directory of the project

\begin{minted}{bash}
source devel/setup.bash
sh scripts/run.sh
\end{minted}

or :

\begin{minted}{bash}
source devel/setup.bash
rosrun map_fusion map_fusion_node
\end{minted}

\section{Use}

\subsection{Set the parameters}

The parameters of the project can be set in the \texttt{src/map\_fusion\_node.cpp} file. In the constructor of the \texttt{EvidentialGrid} class, you can set the following parameters:

\begin{itemize}
    \item \texttt{Map size}: width, height
    \item \texttt{Resolution}: resolution of the map in meters per cell
    \item \texttt{ODOM topic}: the topic where the odometry data is published
    \item \texttt{LIDAR topic}: the topic where the LIDAR laserScan data is published
\end{itemize}

\subsection{Visualize the map}

To visualize the generated map, you can use the \texttt{rviz} tool. Once \texttt{rviz} is started, just add a \texttt{ColorOccupancyGrid} and set the following properties:

\begin{itemize}
    \item \texttt{Topic} : \texttt{/global\_eogm}
    \item \texttt{Voxel coloring Scheme} : \texttt{Cell color}
\end{itemize}

It is recommanded to enable graphics acceleration if you run your system in a virtual machine.