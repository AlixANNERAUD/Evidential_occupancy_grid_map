
\section{Belief Mass Function}

In the Dempster-Shafer theory of evidence, a fundamental concept is that of the belief mass function.
It serves as a mathematical representation of uncertainty associated with propositions within a given frame of discernment.

\subsection{Definition}

The belief mass function $ m $ assigns a numerical value to each subset of the frame of discernment, indicating the degree of belief or support for that subset. Formally, for a frame of discernment $ \Omega $, the belief mass function is defined as:

$$ m: 2^\Omega \rightarrow [0,1] $$

Where:

\begin{itemize}
    \item $ 2^\Omega $ denotes the power set of $ \Omega $, representing all possible subsets of $ \Omega $.
    \item $ m(A) $ represents the belief mass assigned to subset $ A \subseteq \Omega $.
\end{itemize}

In our case of an evidential grid, the frame of discernment is $\Omega = \{F, O, \Omega, \emptyset\}$, where:

\begin{itemize}
    \item $ F $ represents the proposition of unoccupied space.
    \item $ O $ represents the proposition of occupied space.
    \item $ \Omega $ denotes the entire frame of discernment (all possible propositions).
    \item $ \emptyset $ denotes the empty set, representing ignorance or lack of information.
\end{itemize}

In summary, belief mass functions provide a formalism for representing and reasoning with uncertain evidence, making them a key component of the Dempster-Shafer theory and applications such as decision-making under uncertainty and information fusion.


\subsection{Properties}

\begin{itemize}
    \item Normalization: The sum of belief masses assigned to all subsets of $ \Omega $ equals 1:
          $$ \sum_{A \subseteq \Omega} m(A) = 1 $$
    \item Conflict Measure: The belief mass function accounts for conflicts or overlaps between different subsets. This is crucial for combining evidence from multiple sources.
    \item Ignorance Representation : The belief mass function can explicitly model ignorance or lack of information by assigning mass to the empty set $\emptyset$.
          This represents the uncertainty about all subsets of $ \Omega $.
\end{itemize}

\subsection{Interpretation}

\begin{itemize}
    \item Support: High belief mass assigned to a subset indicates strong support or evidence for the propositions within that subset.
    \item Uncertainty: Low belief mass suggests uncertainty or lack of evidence for the propositions in the subset.
\end{itemize}

